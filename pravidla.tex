\sec Pravidla kategorie RES
\rfc {přidat citace na pravidla}
Kategorie F3B-RES (popř. F3RES) je soutěžní třída radiem řízených modelů větroňů s maximálním rozpětím 2 metry a převažující dřevěnou konstrukcí. Od ovládaných funkcí je odvozen název celé kategorie. Rudder, Elevator, Spoiler - Směrovka, Výškovka, Spoiler. Model je do vzduchu vypouštěn pomocí gumicuku. V soutěži se hodnotí časová délka letu a přesnost přistání.


\rfc {lze spoiler nazvat rušič vztlaku? Nebo lépe brzda? Brzdící klapka? Přistávací klapka?}
\rfc {Hodí se název gumicuk do Bakalářské práce? Nebo spíš guma a vlasec?}


\secc Průběh soutěže

Každá soutěž se skládá z alespoň 4 předkol. Každé kolo jsou účastníci rozděleni do více skupin. Výsledky jsou normalizovány tak, aby se snížil faktor změny počasí mezi skupinami. Z těchto předkol se vyberou 4 (maximálně 8) soutěžící s nejvíce body do finálové skupiny. Ta následně odlétá 2 finálová kola tzv. \uv{Fly-Off}, čímž se určí konečné pořadí. Finálová kola mají stejný počet soutěžících jako tomu bylo v předkolech.


\secc Průběh předkol a finálních kol

Model je do vzduchu vytahován pomocí gumicuku RES 100 od firmy EMC Vega. Zpravidla se skládá z 14,7 metrů gumy a 100 metrů nylonového vlasce. V případě malé plochy letiště nebo kvůli zkrácení letových časů může organizátor nařídit zkrácení gumicuku.

Po vystoupání do požadované výšky se model odpoutá a začíná měřený soutěžní let, ve kterém se pilot snaží udržet ve vzduchu co nejdéle, zejména pomocí termických stoupavých proudů. Měření času končí po zastavení modelu na zemi, nebo po uplynutí pracovního okna vyhrazeného pro dané kolo. Maximální doba letu je 360 s v rámci 9 minutového pracovního okna na jedno kolo.

Soutěžní let je zakončený hodnoceným přistáním na bod, který je každému soutěžícímu přiřazen před letem. Není povolené přistání \uv{oštěpem}, tzn. zapíchnutí modelu pod úhlem do země tak, že konec trupu zůstane nad zemí.


\secc Bodování

Letový čas v rámci jednoho kola je zaokrouhlen dolů na celé sekundy. Za každou vteřinu letu získá soutěžící 2 body. 
Následně se vyhodnotí body za přistání. Po zastavení modelu na zemi se změří vzdálenost mezi špičkou trupu letadla a přistávacího bodu. Pokud je tato vzdálenost menší než 0,2 m získává soutěžící plný počet bodů - 100. Bodové ohodnocení se postupně snižuje až do 15 metrů od stanoveného místa přistání. Pokud je vzdálenost větší než 15 metrů, nezískává pilot za přistání žádný bod.

Maximální počet bodů, který lze za jeden soutěžní let je 820. Tedy 720 bodů za plně využitý maximální čas (360 sekund) a 100 bodů za přistání do 0,2 m od stanoveného místa.

Po dolétnutí každé skupiny soutěžících se vyhodnotí výsledky. Pilotovi s největším počtem bodů se za výhru ve skupině přiřadí 1000 bodů. Dalším soutěžícím ze stejné skupiny se body přepočítají poměrným systémem vůči výherci. 

\rfc {je přepočítávání srozumitelné? Nebo je lepší ho zcela vynechat?}


\secc Požadavky na letadlo

Model se obvykle skládá z křídel, trupu a ocasních ploch. Soutěže se mohou účasnit také modely bez trupu či ocasní části, avšak mohou mít ovládané pouze 2 řídící plochy a každá může být řízena pouze jedním servem (například samokřídla).

Maximální povolené rozpětí je 2000 mm. Délka letadla, minimální ani maximální hmotnost nejsou omezeny.

Model s převažující dřevěnou konstrukcí znamená, že může být použito kompozitových dílů pouze na:
\begitems
*zadní část trupu maximálně do půlky hloubky křídla a to ve formě kulaté nebo profilované trubky 
*nosník či spojky křídel ve formě kulaté nebo profilované trubky 
*náběžnou hranu ve formě tyče, trubky (nemůže být použito kompozitových D-boxů ani celých křídel)
\enditems

\rfc {poslední dvě odrážky chybí v pravidlech z 1.1.2017}
Všude jinde musí být použito výhradně dřevo a balsa.

V letadle je možné mít ballast z důvodu lepší pronikavosti proti větru. Zátěž se musí nacházet uvnitř modelu a musí být bezpečně zajištěna.

Spoilerů může být na letadle více a mohou být ovládány jedním nebo dvěma servy. Spoiler se dále musí nacházet na vrchní straně křídla alespoň 5 cm před odtokovou hranou křídla.

\rfc {Ocitovat jednotlivá pravidla - celá přiložit do přílohy}