\chap Úvod
\sec Modely kategorie RES
\secc RESolution

Jedná se o model s celodřevěným trupem s výškovkou a směrovkou klasického uspořádání. Uhlíkové trubky je použito pouze na hlavní nosník křídla, spolu s uhlíkovou tyčkou náběžné hrany. Díky hmotnosti okolo 600 g, a zároveň i zvolenému profilu S 7012 se model dokáže prosadit i v silnějším větru. Je tedy univerzálnější, oproti lehkým letadlům však nedosahuje takových výkonů při klidných podmínkách. \cite [Parek2002]

\midinsert \clabel[RESolution_specifikace] {Specifikace modelu RESolution}
\ctable{ll}{
%\hfil & \crl
Délka (mm)						&	1120		\cr
Rozpětí (mm)					&	1980		\cr
Profil							&	S 7012		\cr
Výška profilu (u kořene)		&	17,5 (8 \%)	\cr
Hloubka křídla - u kořene (mm)	&	187			\cr
Štíhlost 						&	11,3		\cr	
Plocha křídla (dm$^2$)			&	34,8		\cr
Hmotnost (g) 					&	590			\cr
Plošné zatížení (g/dm$^2$)		&	17			\cr 
}
\caption/t Specifikace modelu RESolution.\cite[Schwartz2014]
\endinsert

\secc  PURES V2

Model PURES je osazen uhlíkovou trubkou v zadní části trupu, na které jsou umístěné deskové ocasní plochy bez profilu do V. Jedna brzdící klapka umístěna uprostřed křídla. Stejně jako u modelu RESolution je využita uhlíková trubka jako hlavní nosník. Rozdíl je však v použití dvou různých průměrů (tlustší na střed křídla, tenčí na uši). Křídlo je rozebíratelné na 3 části pro lepší části - středová část a 2 uši. Konstruoval Josef Gergetz (2012-2013). 

\midinsert \clabel[PURES_specifikace] {Specifikace modelu PURES}
\ctable{ll}{
%\hfil & \crl
Délka (mm)						&	1180		\cr
Rozpětí (mm)					&	2000		\cr
Profil							&	AG35-AG36	\cr
Výška profilu (u kořene)		&	13,5 (8 \%)	\cr
Hloubka křídla - u kořene (mm)	&	212			\cr
Štíhlost 						&	11			\cr	
Plocha křídla (dm$^2$)			&	36,5		\cr
Hmotnost (g) 					&	436			\cr
Plošné zatížení (g/dm$^2$)		&	11,9		\cr 
}
\caption/t Specifikace modelu PURES.\cite[Schwartz2014]
\endinsert

\secc Porovnání
Z porovnání jednotlivých modelů je patrné, že plocha modelů kategorie RES se pohybuje v rozmezí 30-37 dm$^2$ a plošné zatížení v rozmezí 12-18 g/dm$^2$. Nejlehčí modely atakují hodnotu 400 g letové hmotnosti. Klasická koncepce uspořádání ocasních ploch (výška+směrovka) převládá u většiny modelů, avšak uspořádání dvojice ploch do V není něco, čemu by se konstruktéři této kategorie vyhýbali.

%\midinsert
%\picw=5cm \cinspic getpic.jpg
%\caption/f Model kategorie RES.
%\endinsert